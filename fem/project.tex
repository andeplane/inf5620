\documentclass[a4paper,10pt]{article}
%\usepackage[latin1]{inputenc}
%\usepackage[english]{babel}
\usepackage{graphicx}
\usepackage{graphics}
\usepackage{enumerate}
\usepackage{fullpage}
\usepackage{amsmath}
\usepackage{amsfonts}
\usepackage{hyperref}
\usepackage{ifsym}
\usepackage{parskip}
\usepackage[numbers,sectionbib]{natbib}

% Egne kommandoer for enklere vektornotasjon.
\renewcommand{\vec}[1]{\mathbf{#1}}
\renewcommand{\(}{\left(}
\renewcommand{\)}{\right)}
%\renewcommand{\>}{\right>}
%\renewcommand{\<}{\left<}
\newcommand{\dm}[1]{\text{d}#1}
\newcommand{\dd}[2]{\frac{\mathrm{d}#1}{\mathrm{d}#2}}
\newcommand{\ddd}[2]{\frac{\mathrm{d^2}#1}{\mathrm{d}#2^2}}
\newcommand{\dpart}[2]{\frac{\partial#1}{\partial#2}}
\newcommand{\dpartt}[2]{\frac{\partial^2#1}{\partial#2^2}}
\newcommand{\qqq}{\qquad\qquad\qquad}
\newcommand{\f}[2]{\frac{#1}{#2}}
\newcommand{\bs}[1]{$\boldsymbol #1$}
\newcommand{\bsa}[1]{\boldsymbol #1}
\usepackage{listings}
\usepackage{color}
\usepackage{textcomp}
\definecolor{listinggray}{gray}{0.9}
\definecolor{lbcolor}{rgb}{0.9,0.9,0.9}
\lstset{
 backgroundcolor=\color{white},
 tabsize=4,
 rulecolor=,
 language=c++, 
 basicstyle=\scriptsize,
 upquote=true,
 aboveskip={1.5\baselineskip},
 columns=fixed,
 showstringspaces=false,
 extendedchars=true,
 breaklines=true,
 prebreak = \raisebox{0ex}[0ex][0ex]{\ensuremath{\hookleftarrow}},
 frame=single,
 showtabs=false,
 showspaces=false,
 showstringspaces=false,
 identifierstyle=\ttfamily,
 keywordstyle=\color[rgb]{0,0,1},
 commentstyle=\color[rgb]{0.133,0.545,0.133},
 stringstyle=\color[rgb]{0.627,0.126,0.941},
}
\usepackage{subfig}
%\usepackage{wrapfig}
\usepackage{epstopdf}

\title{Project 1 - INF5620}

\date{\today}
\author{Kand. Nr. 31}

\newenvironment{changemargin}[2]{%
 \begin{list}{}{%
 %\setlength{\topsep}{0pt}%
 \setlength{\leftmargin}{#1}%
 \setlength{\rightmargin}{#2}%
 %\setlength{\listparindent}{\parindent}%
 %\setlength{\itemindent}{\parindent}%
 %\setlength{\parsep}{\parskip}%
 }%
 \item[]}{\end{list}}
 
 \newcommand{\maxFigure}[4]{
 \begin{figure}[htp!]
 \begin{changemargin}{-3cm}{-1cm}
 \begin{center}
 %\includegravarphics[width=\paperwidth + 3cm,height=\paperheight,keepaspectratio]{#2}
 \includegravarphics[scale = #2]{#3}
 \end{center}
 \end{changemargin}
\vspace{-10pt}
  \caption{\textit{#1} }
  \label{#4}
 \end{figure}
 }
 
 \makeatletter
 \setlength{\abovecaptionskip}{6pt}   % 0.5cm as an example
\setlength{\belowcaptionskip}{6pt}   % 0.5cm as an example
% This does justification (left) of caption.
\long\def\@makecaption#1#2{%
  \vskip\abovecaptionskip
  \sbox\@tempboxa{#1: #2}%
  \ifdim \wd\@tempboxa >\hsize
    #1: #2\par
  \else
    \global \@minipagefalse
    \hb@xt@\hsize{\box\@tempboxa\hfil}%
  \fi
  \vskip\belowcaptionskip}
\makeatother
%%%%%%%%%%%%%%%%%%%%%%%%%%%%%%%%%%%%%%%%%%%%%%%%%%%%%%%%%%%%%
%% DOCUMENT
%%%%%%%%%%%%%%%%%%%%%%%%%%%%%%%%%%%%%%%%%%%%%%%%%%%%%%%%%%%%%
\begin{document}
The 1d wave equation looks like
\begin{align*}
  u_{tt} = c^2 u_{xx}
\end{align*}
\section*{a)}
We use a finite difference scheme in the time dimension
\begin{align*}
  \f{u^{n-1} - 2u^n + u^{n+1}}{\Delta t^2} = c^2u_{xx}
\end{align*}
This can be solved for the next time step
\begin{align*}
  u^{n+1} = \Delta t^2c^2u_{xx} - u^{n-1} + 2u^n
\end{align*}
\section*{c)}
We rearrange the last equation 
\begin{align*}
  u^{n+1} -\Delta t^2c^2u_{xx} + u^{n-1} - 2u^n = \mathcal L(u) = R
\end{align*}
We want $R$ to be orthogonal to the vector space $\mathcal V$, so we multiply by a basis function $\varphi_i$ and integrate
\begin{align*}
  (R,\varphi_i) &= \int_0^L \dm x R\varphi_i\\
  &= \int_0^L \dm x \Big[u^{n+1} - 2u^n + u^{n-1} - c^2\Delta t^2 u''\Big]\varphi_i\\
  &= \int_0^L \dm x u^{n+1}\varphi_i - 2\int_0^L \dm x u^{n}\varphi_i + \int_0^L \dm x u^{n-1}\varphi_i - c^2\Delta t^2\int_0^L \dm x u''\varphi_i,
\end{align*}
where we use $u''$ instead of $u_{xx}$. We look at the second derivative term and apply integration by parts
\begin{align*}
  \int_0^L \dm x u''\varphi_i = u'\varphi_i\Big|^L_0 - \int_0^L \dm x u'\varphi_i' = -\int_0^L \dm x u'\varphi_i' = -(u',\varphi_i'),
\end{align*}
because the boundary terms are zero. We can then write the residual as
\begin{align*}
  (R,\varphi_i) &= (u^{n+1},\varphi_i) - 2(u^{n},\varphi_i) + (u^{n-1},\varphi_i) + C^2(u',\varphi_i') = 0
\end{align*}
where we have defined $C^2=\Delta t^2c^2$, using that the residual should be orthogonal to all basis vectors and evaluating each $u$ at timestep $n$. By inserting $u=\sum_j c_j \varphi_j$, we get
\begin{align*}
  (\sum_j c_j^{n+1}\varphi_j,\varphi_i) &= 2(\sum_j c_j^n\varphi_j,\varphi_i) - (\sum_j c_j^{n-1}\varphi_j,\varphi_i) - C^2(\sum_j c_j^n\varphi_j',\varphi_i').
\end{align*}
Since the P1 elements have the property that $u(x_i) = \sum_j c_j \varphi_j(x_i) = c_j\delta_{ij}$, we have
\begin{align*}
  c_j^{n+1}(\varphi_j,\varphi_i) &= 2c_j^n(\varphi_j,\varphi_i) - c_j^{n-1}(\varphi_j,\varphi_i) + C^2c_j^n(\varphi_j',\varphi_i').
\end{align*}
We recognize this as equation $i$ in the matrix multiplication
\begin{align*}
  Mc^{n+1} &= 2Mc^n - Mc^{n-1} + C^2Kc^n,
\end{align*}
where $M_{ij} = (\varphi_i,\varphi_j)$ and $K_{ij} = (\varphi_i',\varphi_j')$. With the P1 elements, we then have the $5\times5$ $M$ matrix given as
\begin{align*}
  M = \frac{h}{6}\left(
  \begin{array}{c c c c c}
  2 & 1 & 0 & 0 & 0\\
  1 & 4 & 1 & 0 & 0\\
  0 & 1 & 4 & 1 & 0\\
  0 & 0 & 1 & 4 & 1\\
  0 & 0 & 0 & 1 & 2
  \end{array}
  \right)
\end{align*}
and the $5\times5$ $K$ matrix looks like
\begin{align*}
  K = \frac{1}{h}\left(
  \begin{array}{c c c c c}
  2 & -1 & 0 & 0 & 0\\
  -1 & 2 & -1 & 0 & 0\\
  0 & -1 & 2 & -1 & 0\\
  0 & 0 & -1 & 2 & -1\\
  0 & 0 & 0 & -1 & 2
  \end{array}
  \right).
\end{align*}
\begin{align}
  \label{eq:weirdscheme}
  \Big[D_tD_t(u + \f{1}{6}\Delta x^2 D_xD_xu) = c^2D_xD_xu\Big]^n_i
\end{align}
\section*{Stability analysis}
We expand $u$ in a discrete Fourier basis so that
\begin{align}
  \label{eq:fourierbasis}
  u_p^n = \exp\Big[i(kp\Delta x - \tilde \omega n\Delta t)\Big]
\end{align}
Writing equation \eqref{eq:weirdscheme} on discretized form
\begin{align*}
  6C^2\Big[u_{i-1}^n - 2u_i^n + u_{i+1}^n\Big] &=
  6(u_i^{n-1} - 2u_i^n + u_i^{n+1})
   + (
  u_{i-1}^{n-1} - 2u_{i-1}^n + u_{i-1}^{n+1})\\
  &\quad-2(
  u_i^{n-1} - 2u_i^n +u_i^{n+1})
  + (
  u_{i+1}^{n-1} - 2u_{i+1}^n + u_{i+1}^{n+1})\\
  &= 4(u_i^{n-1} - 2u_i^n + u_i^{n+1})
   + (
  u_{i-1}^{n-1} - 2u_{i-1}^n + u_{i-1}^{n+1})\\
  &\quad + (
  u_{i+1}^{n-1} - 2u_{i+1}^n + u_{i+1}^{n+1})
\end{align*}
and inserting \eqref{eq:fourierbasis} into the LHS
\begin{align*}
  L.H.S. &=6C^2e^{-i\tilde\omega n \Delta t}\Big[e^{ik(p-1)\Delta x} -2e^{ikp\Delta x} + e^{ik(p+1)\Delta x}\Big]\\
  &=6C^2e^{i(kp\Delta x - \tilde\omega n \Delta t)}\Big[e^{-ik\Delta x} - 2 + e^{ik\Delta x}\Big]\\
  &=-24C^2e^{i(kp\Delta x - \tilde\omega n \Delta t)}\sin^2\Big(\frac{k\Delta x}{2}\Big)
\end{align*}
and into RHS
\begin{align*}
  R.H.S &= -16e^{i(kp\Delta x - \tilde\omega n \Delta t)}\sin^2\Big(\frac{\tilde \omega\Delta t}{2}\Big) -4e^{i(k(p-1)\Delta x - \tilde\omega n \Delta t)}\sin^2\Big(\frac{\tilde \omega\Delta t}{2}\Big)\\
  &\quad -4e^{i(k(p+1)\Delta x - \tilde\omega n \Delta t)}\sin^2\Big(\frac{\tilde \omega\Delta t}{2}\Big)\\
  &=-4e^{i(kp\Delta x - \tilde\omega n \Delta t)}\sin^2\Big(\frac{\tilde\omega\Delta t}{2}\Big)\Big[e^{-ik\Delta x} + 4 + e^{ik\Delta x}\Big]\\
  &=-8e^{i(kp\Delta x - \tilde\omega n \Delta t)}\sin^2\Big(\frac{\tilde\omega\Delta t}{2}\Big)\Big[\cos\Big(k\Delta x\Big) + 2\Big].
\end{align*}
Combining both sides gives
\begin{align*}
  \sin^2\Big(\frac{\tilde\omega\Delta t}{2}\Big) &= \frac{3C^2}{\cos(k\Delta x) + 2}\sin^2\Big(\frac{k\Delta x}{2}\Big),
\end{align*}
and by square rooting, we get
\begin{align*}
  \sin\Big(\frac{\tilde\omega\Delta t}{2}\Big) &= \frac{\sqrt3C}{\Big[\cos(k\Delta x) + 2\Big]^\frac{1}{2}}\sin\Big(\frac{k\Delta x}{2}\Big).
\end{align*}
The sine on the right hand side has its maximum values $\pm 1$ when $k\Delta x=(n+1)\pi$. The cosine in the denominator is then $\pm 1$. The worst case scenario is if $\cos(k\Delta x)=-1$, where we have the stability criterion $C\sqrt 3 \leq 1$ giving
\begin{align*}
  C \leq \frac{1}{\sqrt 3},
\end{align*}
because the sine on the left hand side must evaluate to less than or equal to 1. The solution -1 will not appear since we actually have a sine squared.
\end{document}