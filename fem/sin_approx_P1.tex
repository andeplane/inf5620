\documentclass[a4paper,10pt]{article}
%\usepackage[latin1]{inputenc}
%\usepackage[english]{babel}
\usepackage{graphicx}
\usepackage{graphics}
\usepackage{enumerate}
\usepackage{fullpage}
\usepackage{amsmath}
\usepackage{amsfonts}
\usepackage{hyperref}
\usepackage{ifsym}
\usepackage{parskip}
\usepackage[numbers,sectionbib]{natbib}

% Egne kommandoer for enklere vektornotasjon
\renewcommand{\vec}[1]{\mathbf{#1}}
\renewcommand{\(}{\left(}
\renewcommand{\)}{\right)}
%\renewcommand{\>}{\right>}
%\renewcommand{\<}{\left<}
\newcommand{\dm}[1]{\text{d}#1}
\newcommand{\dd}[2]{\frac{\mathrm{d}#1}{\mathrm{d}#2}}
\newcommand{\ddd}[2]{\frac{\mathrm{d^2}#1}{\mathrm{d}#2^2}}
\newcommand{\dpart}[2]{\frac{\partial#1}{\partial#2}}
\newcommand{\dpartt}[2]{\frac{\partial^2#1}{\partial#2^2}}
\newcommand{\qqq}{\qquad\qquad\qquad}
\newcommand{\f}[2]{\frac{#1}{#2}}
\newcommand{\bs}[1]{$\boldsymbol #1$}
\newcommand{\bsa}[1]{\boldsymbol #1}
\usepackage{listings}
\usepackage{color}
\usepackage{textcomp}
\definecolor{listinggray}{gray}{0.9}
\definecolor{lbcolor}{rgb}{0.9,0.9,0.9}
\lstset{
 backgroundcolor=\color{white},
 tabsize=4,
 rulecolor=,
 language=c++, 
 basicstyle=\scriptsize,
 upquote=true,
 aboveskip={1.5\baselineskip},
 columns=fixed,
 showstringspaces=false,
 extendedchars=true,
 breaklines=true,
 prebreak = \raisebox{0ex}[0ex][0ex]{\ensuremath{\hookleftarrow}},
 frame=single,
 showtabs=false,
 showspaces=false,
 showstringspaces=false,
 identifierstyle=\ttfamily,
 keywordstyle=\color[rgb]{0,0,1},
 commentstyle=\color[rgb]{0.133,0.545,0.133},
 stringstyle=\color[rgb]{0.627,0.126,0.941},
}
\usepackage{subfig}
%\usepackage{wrapfig}
\usepackage{epstopdf}

\title{Project 1 - INF5620}

\date{\today}
\author{Kand. Nr. 31}

\newenvironment{changemargin}[2]{%
 \begin{list}{}{%
 %\setlength{\topsep}{0pt}%
 \setlength{\leftmargin}{#1}%
 \setlength{\rightmargin}{#2}%
 %\setlength{\listparindent}{\parindent}%
 %\setlength{\itemindent}{\parindent}%
 %\setlength{\parsep}{\parskip}%
 }%
 \item[]}{\end{list}}
 
 \newcommand{\maxFigure}[4]{
 \begin{figure}[htp!]
 \begin{changemargin}{-3cm}{-1cm}
 \begin{center}
 %\includegraphics[width=\paperwidth + 3cm,height=\paperheight,keepaspectratio]{#2}
 \includegraphics[scale = #2]{#3}
 \end{center}
 \end{changemargin}
\vspace{-10pt}
  \caption{\textit{#1} }
  \label{#4}
 \end{figure}
 }
 
 \makeatletter
 \setlength{\abovecaptionskip}{6pt}   % 0.5cm as an example
\setlength{\belowcaptionskip}{6pt}   % 0.5cm as an example
% This does justification (left) of caption.
\long\def\@makecaption#1#2{%
  \vskip\abovecaptionskip
  \sbox\@tempboxa{#1: #2}%
  \ifdim \wd\@tempboxa >\hsize
    #1: #2\par
  \else
    \global \@minipagefalse
    \hb@xt@\hsize{\box\@tempboxa\hfil}%
  \fi
  \vskip\belowcaptionskip}
\makeatother
%%%%%%%%%%%%%%%%%%%%%%%%%%%%%%%%%%%%%%%%%%%%%%%%%%%%%%%%%%%%%
%% DOCUMENT
%%%%%%%%%%%%%%%%%%%%%%%%%%%%%%%%%%%%%%%%%%%%%%%%%%%%%%%%%%%%%
\begin{document}
\section*{Exercise 10}
In this exercise we will look at a finite element approximation of the function $f(x)=\sin(x)$. We will work in the domain $\Omega=[0,\pi]$ and use two P1 elements, each if size $\pi/2$. The local P1 elements are given by the Lagrange polynomials
\begin{align*}
  \tilde\phi_0(X) = \frac{1}{2}(1-X)\\
  \tilde\phi_1(X) = \frac{1}{2}(1+X)
\end{align*}
The affine transform is given by
\begin{align*}
  x = x_m + \frac{1}{2}hX
\end{align*}
where $x_m$ is the midpoint in the interval and $h=\pi/2$. We have the midpoints for the elements $x_m^{e=1}=\pi/4$ and $x_m^{e=2}=3\pi/4$. The jacobi determinants are the same, $\det J = h/2 = \pi/4$. Being general in mind, keeping the $h$ factor, the coefficient matrix is given as
\begin{align*}
  \tilde A^(e)_{0,0} &= \int_{-1}^1  \dm X \det J \tilde\phi_0(X)\tilde\phi_0(X)\\
  &= \frac{h}{2} \int_{-1}^1 \dm X \frac{1}{2}(1-X)\frac{1}{2}(1-X)\\
  &= \frac{h}{8} \int_{-1}^1 \dm X (1-X)^2 = \frac{h}{24} \Big[(1-X)^3\Big]^1_{-1} = \frac{h}{3}.
\end{align*}
Similarly we get that
\begin{align*}
  \tilde A^(e)_{1,0} &= A^(e)_{0,1} = \frac{h}{6}\\
  \tilde A^(e)_{1,1} &= A^(e)_{0,0} = \frac{h}{3},
\end{align*}
so it finally looks like
\begin{align*}
M' = \left( \begin{array}{cc}
\frac{h}{3} & \frac{h}{6}\\
\frac{h}{6} & \frac{h}{3}
\end{array}
\right).
\end{align*}
The right hand side, the vector $b$, is given as
\begin{align*}
  \tilde b_0^{(e)} &= \int_{-1}^1 \dm X f(x(X))\tilde\phi_0(X) \frac{h}{2} \\
  &= \frac{h}{4}\int_{-1}^1 \dm X \sin(x_m^{e} + \frac{1}{2}hX)(1-X)\\
  &= 
\end{align*}
\end{document}
