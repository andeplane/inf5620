\subsection{Stationary solution - Poiseuilles law}
For a fluid moving in the x-direction, the viscosity coefficient is defined as
\begin{align*}
  \frac{F}{A} = \tau = \mu\left|\dpart{u_x}{y}\right|,
\end{align*}
a proportional factor that describes the relation between shear stress and the velocity gradient normal to the fluid velocity. If we look at a box of height $h$, length $L$ and depth $d$ (we will assume that we are far away from the walls in the depth-direction), the force in the above equation can be written as
\begin{align*}
  F = \mu A \left|\frac{\dm u_x}{\dm y}\right| = \mu Ld\left|\frac{\dm u_x}{\dm y}\right|.
\end{align*}
We define $y=0$ to be at height $h/2$, in the middle of the box. We assume stationary flow, and this force must exact cancel the force made by a pressure difference at $x=0$ and $x=L$, so that
\begin{align*}
  (P_A - P_B)yd = -\mu Ld\left|\frac{\dm u_x}{\dm y}\right|.
\end{align*}
By integrating, we get
\begin{align*}
  -\int_v^0 \dm u_x  &= \frac{(P_A - P_B)}{\mu L}\int_y^{h/2}y'\dm y'\\
  v(y) &= \frac{(P_A - P_B)}{2\mu L}\left[\frac{h^2}{4}-y^2\right],
\end{align*}
which gives the maximum velocity
\begin{align}
  \label{eq:max_vel}
  v_{max} = v(0) = \frac{(P_A - P_B)}{8\mu L}h^2.
\end{align}
\subsection{Numerical results}
An implementation of the Poisuilles law problem can be found in appendix \ref{ap:a}. We will compare the theoretical maximum velocity $v(0)$ with the numerics. Our adjustable parameters is
\begin{itemize}
\item $\Delta x$ - grid size
\item $\Delta P$ - pressure difference
\item $\mu$ - viscosity
\item $L$ - distance between $A$ and $B$ in pressure difference
\item $h$ - tube height
\end{itemize}
We apply equation \eqref{eq:max_vel} with different parameters and run FEniCS until we reach a steady state. The results can be found in table \ref{tab:faen}.

\begin{table}[hb]
  \label{tab:faen}
  \begin{center}
    \begin{tabular}[width=4in]{|c|c|c|c|c|c|c|c|}
      \hline
      \# & $\Delta p$ & $\mu$ & $L$ & $h$ & Analytical & FEniCS & Error\\ \hline
      1 & 1 & 1 & 1 & 1 & 0.125 & 0.125 & 0.000\\
      2 & 2 & 1 & 1 & 1 & 0.250 & 0.250 & 0.000\\
      3 & 1 & 1 & 1 & 2 & 0.500 & 0.500 & 0.000\\
      \hline
    \end{tabular}
  \caption{Comparison of the analytical and the numerical solution of the pressure driven stationary system.}
  \end{center}

\end{table}