\documentclass[a4paper,10pt]{article}
%\usepackage[latin1]{inputenc}
%\usepackage[english]{babel}
\usepackage{graphicx}
\usepackage{graphics}
\usepackage{enumerate}
\usepackage{fullpage}
\usepackage{amsmath}
\usepackage{amsfonts}
\usepackage{hyperref}
\usepackage{ifsym}
\usepackage{parskip}
\usepackage[numbers,sectionbib]{natbib}

% Egne kommandoer for enklere vektornotasjon
\renewcommand{\vec}[1]{\mathbf{#1}}
\renewcommand{\(}{\left(}
\renewcommand{\)}{\right)}
%\renewcommand{\>}{\right>}
%\renewcommand{\<}{\left<}
\newcommand{\dm}[1]{\text{d}#1}
\newcommand{\dd}[2]{\frac{\mathrm{d}#1}{\mathrm{d}#2}}
\newcommand{\ddd}[2]{\frac{\mathrm{d^2}#1}{\mathrm{d}#2^2}}
\newcommand{\dpart}[2]{\frac{\partial#1}{\partial#2}}
\newcommand{\dpartt}[2]{\frac{\partial^2#1}{\partial#2^2}}

\usepackage{listings}
\usepackage{color}
\usepackage{textcomp}
\definecolor{listinggray}{gray}{0.9}
\definecolor{lbcolor}{rgb}{0.9,0.9,0.9}
\lstset{
 backgroundcolor=\color{white},
 tabsize=4,
 rulecolor=,
 language=c++, 
 basicstyle=\scriptsize,
 upquote=true,
 aboveskip={1.5\baselineskip},
 columns=fixed,
 showstringspaces=false,
 extendedchars=true,
 breaklines=true,
 prebreak = \raisebox{0ex}[0ex][0ex]{\ensuremath{\hookleftarrow}},
 frame=single,
 showtabs=false,
 showspaces=false,
 showstringspaces=false,
 identifierstyle=\ttfamily,
 keywordstyle=\color[rgb]{0,0,1},
 commentstyle=\color[rgb]{0.133,0.545,0.133},
 stringstyle=\color[rgb]{0.627,0.126,0.941},
}
\usepackage{subfig}
%\usepackage{wrapfig}
\usepackage{epstopdf}

\title{Project 1 - INF5620}

\date{\today}
\author{Kand. Nr. 31}

\newenvironment{changemargin}[2]{%
 \begin{list}{}{%
 %\setlength{\topsep}{0pt}%
 \setlength{\leftmargin}{#1}%
 \setlength{\rightmargin}{#2}%
 %\setlength{\listparindent}{\parindent}%
 %\setlength{\itemindent}{\parindent}%
 %\setlength{\parsep}{\parskip}%
 }%
 \item[]}{\end{list}}
 
 \newcommand{\maxFigure}[4]{
 \begin{figure}[htp!]
 \begin{changemargin}{-3cm}{-1cm}
 \begin{center}
 %\includegraphics[width=\paperwidth + 3cm,height=\paperheight,keepaspectratio]{#2}
 \includegraphics[scale = #2]{#3}
 \end{center}
 \end{changemargin}
\vspace{-10pt}
  \caption{\textit{#1} }
  \label{#4}
 \end{figure}
 }
 
 \makeatletter
 \setlength{\abovecaptionskip}{6pt}   % 0.5cm as an example
\setlength{\belowcaptionskip}{6pt}   % 0.5cm as an example
% This does justification (left) of caption.
\long\def\@makecaption#1#2{%
  \vskip\abovecaptionskip
  \sbox\@tempboxa{#1: #2}%
  \ifdim \wd\@tempboxa >\hsize
    #1: #2\par
  \else
    \global \@minipagefalse
    \hb@xt@\hsize{\box\@tempboxa\hfil}%
  \fi
  \vskip\belowcaptionskip}
\makeatother
%%%%%%%%%%%%%%%%%%%%%%%%%%%%%%%%%%%%%%%%%%%%%%%%%%%%%%%%%%%%%
%% DOCUMENT
%%%%%%%%%%%%%%%%%%%%%%%%%%%%%%%%%%%%%%%%%%%%%%%%%%%%%%%%%%%%%
\begin{document}
\section*{The material derivative}
\begin{align*}
  \frac{D}{Dt} = \dpart{}{t} + \vec v \cdot \nabla
\end{align*}
\section*{Conservation laws}
Consider a property $L$ that is measureable within a finite volume $\Omega$ with boundary $\partial \Omega$. The rate of change equals the amount that is created/consumed inside the volume or what flows in/out through the boundary. This can be expressed as
\begin{align*}
  \frac{\dm }{\dm t} \int_\Omega L \dm V = -\int_{\partial \Omega} L\vec v \cdot \vec n \dm A - \int_\Omega Q \dm V,
\end{align*}
where $\vec n$ is the normal vector to the boundary pointing outwards, $\vec v$ is the fluid velocity and $Q$ represents the sources or sinks inside the fluid. If we apply the divergence theorem, this becomes
\begin{align*}
  \frac{\dm }{\dm t} \int_\Omega L \dm V = -\int_{\Omega} \nabla \cdot (L\vec v) \dm V - \int_\Omega Q \dm V,
\end{align*}
or simply
\begin{align*}
\int_\Omega \Bigg(\dpart{L}{t} + \nabla \cdot (L\vec v) + Q \Bigg)\dm V = 0.
\end{align*}
But since the volume $\Omega$ is arbitrary chosen, the integrand must be zero
\begin{align*}
  \dpart{L}{t} + \nabla \cdot (L\vec v) + Q = 0.
\end{align*}
\section*{Conservation of momentum}
By using $L=\rho \vec v$, we get the conservation given as
\begin{align*}
  \dpart{\rho\vec v}{t} + \nabla \cdot (\rho\vec v\vec v) + \vec Q = 0.
\end{align*}
The $Q$ term, the source or sink, is simply the body force, i.e. external forces that we denote by $\vec b$. The term with $\vec v\vec v$ is a dyad, i.e. a second rank tensor. This then becomes
\begin{align*}
  \vec v\dpart{\rho}{t} + \rho\dpart{\vec v}{t} + \vec v\vec v\cdot\nabla \rho + \rho\vec v\nabla\cdot \vec v + \rho\vec v\cdot \nabla \vec v = \vec b.
\end{align*}
\end{document}
