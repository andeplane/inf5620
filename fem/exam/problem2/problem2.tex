\documentclass[a4paper,10pt]{article}
%\usepackage[latin1]{inputenc}
%\usepackage[english]{babel}
\usepackage{graphicx}
\usepackage{graphics}
\usepackage{bbm}
\usepackage{enumerate}
\usepackage{fullpage}
\usepackage{amsmath}
\usepackage{amsfonts}
\usepackage{hyperref}
\usepackage{ifsym}
\usepackage{parskip}
\usepackage[numbers,sectionbib]{natbib}

% Egne kommandoer for enklere vektornotasjon
\renewcommand{\vec}[1]{\mathbf{#1}}
\renewcommand{\(}{\left(}
\renewcommand{\)}{\right)}
%\renewcommand{\>}{\right>}
%\renewcommand{\<}{\left<}
\newcommand{\dm}[1]{\text{d}#1}
\newcommand{\dd}[2]{\frac{\mathrm{d}#1}{\mathrm{d}#2}}
\newcommand{\ddd}[2]{\frac{\mathrm{d^2}#1}{\mathrm{d}#2^2}}
\newcommand{\dpart}[2]{\frac{\partial#1}{\partial#2}}
\newcommand{\dpartt}[2]{\frac{\partial^2#1}{\partial#2^2}}

\usepackage{listings}
\usepackage{color}
\usepackage{textcomp}
\definecolor{listinggray}{gray}{0.9}
\definecolor{lbcolor}{rgb}{0.9,0.9,0.9}
\lstset{
 backgroundcolor=\color{white},
 tabsize=4,
 rulecolor=,
 language=c++, 
 basicstyle=\scriptsize,
 upquote=true,
 aboveskip={1.5\baselineskip},
 columns=fixed,
 showstringspaces=false,
 extendedchars=true,
 breaklines=true,
 prebreak = \raisebox{0ex}[0ex][0ex]{\ensuremath{\hookleftarrow}},
 frame=single,
 showtabs=false,
 showspaces=false,
 showstringspaces=false,
 identifierstyle=\ttfamily,
 keywordstyle=\color[rgb]{0,0,1},
 commentstyle=\color[rgb]{0.133,0.545,0.133},
 stringstyle=\color[rgb]{0.627,0.126,0.941},
}
\usepackage{subfig}
%\usepackage{wrapfig}
\usepackage{epstopdf}

\title{Problem 1 - INF5620}

\date{\today}
\author{Anders Hafreager}

\newenvironment{changemargin}[2]{%
 \begin{list}{}{%
 %\setlength{\topsep}{0pt}%
 \setlength{\leftmargin}{#1}%
 \setlength{\rightmargin}{#2}%
 %\setlength{\listparindent}{\parindent}%
 %\setlength{\itemindent}{\parindent}%
 %\setlength{\parsep}{\parskip}%
 }%
 \item[]}{\end{list}}
 
 \newcommand{\maxFigure}[4]{
 \begin{figure}[htp!]
 \begin{changemargin}{-3cm}{-1cm}
 \begin{center}
 %\includegraphics[width=\paperwidth + 3cm,height=\paperheight,keepaspectratio]{#2}
 \includegraphics[scale = #2]{#3}
 \end{center}
 \end{changemargin}
\vspace{-10pt}
  \caption{\textit{#1} }
  \label{#4}
 \end{figure}
 }
 
 \makeatletter
 \setlength{\abovecaptionskip}{6pt}   % 0.5cm as an example
\setlength{\belowcaptionskip}{6pt}   % 0.5cm as an example
% This does justification (left) of caption.
\long\def\@makecaption#1#2{%
  \vskip\abovecaptionskip
  \sbox\@tempboxa{#1: #2}%
  \ifdim \wd\@tempboxa >\hsize
    #1: #2\par
  \else
    \global \@minipagefalse
    \hb@xt@\hsize{\box\@tempboxa\hfil}%
  \fi
  \vskip\belowcaptionskip}
\makeatother
%%%%%%%%%%%%%%%%%%%%%%%%%%%%%%%%%%%%%%%%%%%%%%%%%%%%%%%%%%%%%
%% DOCUMENT
%%%%%%%%%%%%%%%%%%%%%%%%%%%%%%%%%%%%%%%%%%%%%%%%%%%%%%%%%%%%%
\begin{document}
We look at heat conduction in a body with density $\rho$, heat capacity $c$ and heat conduction coefficient $\alpha$. The equation for the temperature distribution $T(x,y,z,t)$ can be written as
\begin{align*}
  \rho c T_t = \nabla \cdot (\alpha(x,y,z)\nabla T).
\end{align*}
The body is a cylinder of length L with isolated cylindrical surface such that $-\alpha\partial_n T=0$ here. The ends are kept at constant temperatures $T_0$ and $T_1$. The left half of the cylinder is made of a material with heat capacity $c_0$ and conduction coefficient $\alpha_0$, while the right half has values $c_1$ and $\alpha_1$. At first, these are two separate cylinders, but at $t=0$ the cylinders are brought together.

\section{Show that the simplification T=T(x,t) is possible in the described problem, where x is a coordinate along the cylinder (just insert T(x,t) in the original problem and see that it fulfills all equations). Set up the simplified PDE with proper boundary and initial conditions.}
We assume that the temperature is only dependent of $x$, so that $T(x,y,z,t) = T(x,t)$. We insert into the diffusion equation
\begin{align*}
  \rho c \dpart{T(x,t)}{t} = \nabla \cdot (\alpha(x,y,z)\nabla T) = \nabla\alpha\cdot\nabla T + \alpha \nabla^2 T
\end{align*}
Since $T$ is only dependent of $x$, this reduces to
\begin{align*}
  \rho c \dpart{T}{t} = \nabla\alpha\cdot\nabla T + \alpha \nabla^2 T = \dpart{\alpha}{x}\dpart{T}{x} + \alpha\dpartt{T}{x} = \dpart{}{x}\left(\alpha\dpart{T}{x}\right)
\end{align*}
with 
\begin{align*}
  \alpha(x) &= \left\{ 
  \begin{array}{l l}
    \alpha_0 & \quad x<L/2\\
    \alpha_1 & \quad x\geq L/2
  \end{array} \right.\\
  c(x) &= \left\{ 
  \begin{array}{l l}
    c_0 & \quad x<L/2\\
    c_1 & \quad x\geq L/2
  \end{array} \right.\\
  T(x,0) &= \left\{ 
  \begin{array}{l l}
    T_0 & \quad x<L/2\\
    T_1 & \quad x\geq L/2
  \end{array} \right.\\
  T(0) &= T_0\\
  T(L) &= T_1
\end{align*}
\section{The 1D PDE problem is discretized by the Forward Euler, Backward Euler, and Crank-Nicolson schemes. Derive the discrete equations for one of these schemes.}
We discretize in the usual way
\begin{align*}
  T(x,t)\approx T(i\Delta x,n\Delta t) \equiv T^n_i
\end{align*}
\subsection{Forward Euler}
\begin{align*}
  \Big[\rho c D_t^+T = D_x\alpha[D_xT]\Big]^n_i
\end{align*}
where we have used the centered difference scheme in the spatial part of the equation. Written out with indices, this becomes
\begin{align*}
  \rho c {T^{n+1}_i - T^{n}_i\over \Delta t} = \frac{1}{\Delta x^2}\Big[\alpha_{i+1/2}\big(T^n_{i+1} - T^n_{i}\big) - \alpha_{i-1/2}\big(T^n_i - T^n_{i-1}\big)\Big]
\end{align*}
which can be solved directly for $T^{n+1}_i$
  \begin{align*}
  T^{n+1}_i = C\Big[\alpha_{i+1/2}\big(T^n_{i+1} - T^n_{i}\big) - \alpha_{i-1/2}\big(T^n_i - T^n_{i-1}\big)\Big] + T^{n}_i
\end{align*}
where $C=\frac{\Delta t}{\rho c\Delta x^2}$.
\section{Assume for simplicity that $c_0=c_1$ and that $\alpha_0=\alpha_1$. With a discontinuous initial conditions, numerical artifacts may appear in the solutions with the three methods. Illustrate such artifacts. A suitable program to play around with is demo\_osc.py.}

\section{Present the ideas and results of an analysis that can explain the artifacts in the previous subproblem.}
When $\alpha$ is a constant, the equation is reduced to
\begin{align*}
  u_t = Cu_xx
\end{align*}



\end{document}
