\documentclass[a4paper,10pt]{article}
%\usepackage[latin1]{inputenc}
%\usepackage[english]{babel}
\usepackage{graphicx}
\usepackage{graphics}
\usepackage{bbm}
\usepackage{enumerate}
\usepackage{fullpage}
\usepackage{amsmath}
\usepackage{amsfonts}
\usepackage{hyperref}
\usepackage{ifsym}
\usepackage{parskip}
\usepackage[numbers,sectionbib]{natbib}

% Egne kommandoer for enklere vektornotasjon
\renewcommand{\vec}[1]{\mathbf{#1}}
\renewcommand{\(}{\left(}
\renewcommand{\)}{\right)}
%\renewcommand{\>}{\right>}
%\renewcommand{\<}{\left<}
\newcommand{\dm}[1]{\text{d}#1}
\newcommand{\dd}[2]{\frac{\mathrm{d}#1}{\mathrm{d}#2}}
\newcommand{\ddd}[2]{\frac{\mathrm{d^2}#1}{\mathrm{d}#2^2}}
\newcommand{\dpart}[2]{\frac{\partial#1}{\partial#2}}
\newcommand{\dpartt}[2]{\frac{\partial^2#1}{\partial#2^2}}

\usepackage{listings}
\usepackage{color}
\usepackage{textcomp}
\definecolor{listinggray}{gray}{0.9}
\definecolor{lbcolor}{rgb}{0.9,0.9,0.9}
\lstset{
 backgroundcolor=\color{white},
 tabsize=4,
 rulecolor=,
 language=c++, 
 basicstyle=\scriptsize,
 upquote=true,
 aboveskip={1.5\baselineskip},
 columns=fixed,
 showstringspaces=false,
 extendedchars=true,
 breaklines=true,
 prebreak = \raisebox{0ex}[0ex][0ex]{\ensuremath{\hookleftarrow}},
 frame=single,
 showtabs=false,
 showspaces=false,
 showstringspaces=false,
 identifierstyle=\ttfamily,
 keywordstyle=\color[rgb]{0,0,1},
 commentstyle=\color[rgb]{0.133,0.545,0.133},
 stringstyle=\color[rgb]{0.627,0.126,0.941},
}
\usepackage{subfig}
%\usepackage{wrapfig}
\usepackage{epstopdf}

\title{Problem 1 - INF5620}

\date{\today}
\author{Anders Hafreager}

\newenvironment{changemargin}[2]{%
 \begin{list}{}{%
 %\setlength{\topsep}{0pt}%
 \setlength{\leftmargin}{#1}%
 \setlength{\rightmargin}{#2}%
 %\setlength{\listparindent}{\parindent}%
 %\setlength{\itemindent}{\parindent}%
 %\setlength{\parsep}{\parskip}%
 }%
 \item[]}{\end{list}}
 
 \newcommand{\maxFigure}[4]{
 \begin{figure}[htp!]
 \begin{changemargin}{-3cm}{-1cm}
 \begin{center}
 %\includegraphics[width=\paperwidth + 3cm,height=\paperheight,keepaspectratio]{#2}
 \includegraphics[scale = #2]{#3}
 \end{center}
 \end{changemargin}
\vspace{-10pt}
  \caption{\textit{#1} }
  \label{#4}
 \end{figure}
 }
 
 \makeatletter
 \setlength{\abovecaptionskip}{6pt}   % 0.5cm as an example
\setlength{\belowcaptionskip}{6pt}   % 0.5cm as an example
% This does justification (left) of caption.
\long\def\@makecaption#1#2{%
  \vskip\abovecaptionskip
  \sbox\@tempboxa{#1: #2}%
  \ifdim \wd\@tempboxa >\hsize
    #1: #2\par
  \else
    \global \@minipagefalse
    \hb@xt@\hsize{\box\@tempboxa\hfil}%
  \fi
  \vskip\belowcaptionskip}
\makeatother
%%%%%%%%%%%%%%%%%%%%%%%%%%%%%%%%%%%%%%%%%%%%%%%%%%%%%%%%%%%%%
%% DOCUMENT
%%%%%%%%%%%%%%%%%%%%%%%%%%%%%%%%%%%%%%%%%%%%%%%%%%%%%%%%%%%%%
\begin{document}
\section{Explain briefly how this equation arises from basic principles in physics and what the individual terms model.}
We look at a sphere with radius $r$, volume $V$ and density $\rho$ falling with gravity $g$ in a viscous fluid with dynamic viscosity $\mu$, density $\rho_f$. The forces are divided into three parts
\begin{align*}
  F_g &= -mg = -V\rho g\\
  F_d &= -6\pi r\mu v\\
  F_f &= V\rho_f g
\end{align*}
where $F_G = F_g + F_f = gV(\rho_f - \rho)$ is the total force due to gravity including the uplift from archimedes principle. Using Newtons 2nd law, we have
\begin{align}
  \nonumber
  V\rho \dot v &= -V\rho g -6\pi r\mu v + V\rho_f g\\
  \nonumber
  &= gV(\rho_f - \rho) -6\pi r\mu \cdot v\\
  \label{eq:newton2nd}
  \dot v &= {g(\rho_f - \rho) \over \rho} -{6\pi r\mu \over V\rho }v = A + Bv,
\end{align}
where $A=g(\rho_f - \rho)/\rho$ and $B=-6\pi r\mu /V\rho$. This is a first order differential equation for the velocity $v$. Note that the friction term is proportional to the radius and the viscosity of the fluid. This term is called Stokes' law and can be derived by solving the Navier-Stokes equations assuming Stokes flow. Stokes flow has low Reynold numbers, and this is also a requirement for \eqref{eq:newton2nd} to hold. The reynold number is defined as
\begin{align}
  \label{eq:reynold}
  \text{Re} = {\rho d|v| \over \mu},
 \end{align}
where $d$ is a characteristic length of the object, i.e. the diameter. The thermal velocity is given when the acceleration is zero
\begin{align*}
  v_{max} = -{A \over B} = {Vg(\rho_f - \rho) \over {6\pi r\mu}},
\end{align*}
whereas the full analytical solution to the differential equation is
\begin{align*}
  v(t) = c_1\exp(Bt) - \frac{A}{B}.
\end{align*}
If we insert the initial condition $v(0)=0$, we have
\begin{align*}
  v(t) = {Vg(\rho_f - \rho) \over {6\pi r\mu}}\left[1 - \exp\left(-\frac{6\pi r\mu}{V\rho}t\right)\right].
\end{align*}

\section{Derive a Forward Euler, Backward Euler, and a Crank-Nicolson scheme for the equation. Mention other possible schemes too.}
In order to solve this system numerically, we have to discretize the time dimension so that
\begin{align*}
  t = n\Delta t
\end{align*}
and our numerical solutions obeys
\begin{align*}
  v(t) \approx v(n\Delta t) \equiv v^n
\end{align*}
The only way to evolve the system in time is by connecting $v_n$ to $v^{n+1}$ by using a scheme for the time derivative. 
\subsection{Operator notation}
We use the operator notation to describe the different schemes, where the differential equation
\begin{align*}
  \dpart{u}{t} = f(u)
\end{align*}
is discretized as
\begin{alignat}{3}
  &[D^+_tv = f(v)]^n & \qquad \quad \text{Forward Euler}\\
  &[D^-_tv = f(v)]^n & \qquad \quad \text{Backward Euler}\\
  &[D_tv = f(v)]^{n+1/2} & \qquad \quad \text{Crank-Nicolson}
\end{alignat}
where we'll investigate these in detail in the following sections.

\subsection{Forward Euler}
The FE scheme is written with operator notation
\begin{align*}
  [D^+_tv = A + Bv]^n = {v^{n+1} - v^n \over \Delta t} = A + Bv^n
\end{align*}
which can be solved directly to give an explicit scheme for $v^{n+1}$
\begin{align}
  \label{eq:forward_euler}
  v^{n+1} = \Delta t(A + Bv^n) + v^n = v^n(1 + \Delta t B) + \Delta tA
\end{align}
\subsection{Backward Euler}
If we instead evaluate the derivative in timestep $n+1$, we get the Backward Euler scheme
\begin{align*}
  [D^-_tv = A + Bv]^n = {v^{n} - v^{n-1} \over \Delta t} = A + Bv^n
\end{align*}
which also gives an explicit scheme
\begin{align*}
  v^n(1 - \Delta t B) &= \Delta t A + v^{n-1}\\
  v^n &= {\Delta t A + v^{n-1} \over 1 - \Delta t B}
\end{align*}
\subsection{Crank-Nicolson}
In the Crank-Nicolson scheme, we expand the function around $n+1/2$, so that
\begin{align*}
  [D_tv = f(v)]^{n+1/2} &= {v^{n+1/2 + 1/2} - v^{n+1/2 - 1/2} \over 2\Delta t/2} = {v^{n+1} - v^{n} \over \Delta t}\\
  &= f(v^{n+1/2}) = A + Bv^{n+1/2},
\end{align*}
where we have to find a way to evalute $v^{n+1/2}$. A usual approach is to use the arithmetic mean $v^{n+1/2} = 1/2(v^{n+1} + v^n)$ which again will give us an explicit scheme for $v^{n+1}$
\begin{align*}
  {v^{n+1} - v^{n} \over \Delta t} &= A + \frac{B}{2}(v^{n+1} + v^n)\\
  v^{n+1} &= {A\Delta t + v^n\left(1 + \frac{B\Delta t}{2}\right) \over \left(1 - \frac{B\Delta t}{2}\right)}
\end{align*}
\subsection{$\theta$-rule}
All these can be summarized into one scheme with a parameter $\theta$ determining which scheme we end up with. We define $f(v^{n+\theta})=f(\theta v^{n+1} + (1-\theta)v^n)$
\begin{align*}
  {v^{n+1} - v^n \over \Delta t} = A + B\left(\theta v^{n+1} + (1-\theta)v^n\right)
\end{align*}
which we can solve for $u^{n+1}$
\begin{align}
  \label{eq:theta}
  v^{n+1} &= {A\Delta t + v^n\left[1 + B\Delta t(1-\theta) \right] \over 1 - \theta B\Delta t}
\end{align}
which will give Forward Euler, Crank-Nicolson and Backward Euler for $\theta \{0,\frac{1}{2},1\}$ respectivly.
\subsection{Runge-Kutta}

\section{Illustrate what kind of numerical artifacts that may appear when using the Forward Euler, Backward Euler, and a Crank-Nicolson schemes. Explain the reason for the artifacts (motivated by a mathematical analysis of the schemes).}
\subsection{Oscillations and divergence}
If we start with the initial velocity $v_0$ and use \eqref{eq:theta}, we can find $v_1$, $v_2$ etc
\begin{align*}
  v^1 &= {A\Delta t + v^0\left[1 + B\Delta t(1-\theta) \right] \over 1 - \theta B\Delta t} = v^0C + D\\
  v_2 &= v_1C + D = (v_0C + D)C + D = v_0C^2 + DC + D\\
  v_3 &= v_2C + D = (v_1C + D)C + D = ((v_0C + D)C + D)C + D\\
      &= v_0C^3 + DC^2 + DC + D.
\end{align*}
where
\begin{align*}
  C&={1 + B\Delta t(1-\theta)\over 1 - \theta B\Delta t}\\  
  D&={A\Delta t \over 1 - \theta B\Delta t}.
\end{align*}
We recognize a pattern and can write the $n$'th term as
\begin{align*}
  v_n = v_0 C^n + D\sum_{k=0}^{n-1}C^k = v_0C^n + D\frac{1 - C^n}{1 - C}
\end{align*}
where we have used the sum of a geometric series. In order to have a convergent series, we require $|C|<1$, which gives a stability criterion for $\Delta t$. If we also want non-oscillating solutions (which are non-physical), we require $0 < C < 1$
\begin{align*}
  0 &< C\\
  0 &< {1 + B\Delta t(1-\theta)\over 1 - \theta B\Delta t}.
\end{align*}
Since $B$ is negative, the denominator will always be greater than 1, so this inequality reduces to
\begin{align}
  \label{eq:backward_stable}
  1 + B\Delta t(1-\theta) \geq 0\\
  \nonumber
  \Delta t \leq -{1 \over B(1-\theta)}
\end{align}
Note that \eqref{eq:backward_stable} always holds for Backward Euler, i.e. $\theta = 1$. The other inequality gives
\begin{align*}
  C &\leq 1\\
  {1 + B\Delta t(1-\theta)\over 1 - \theta B\Delta t} &\leq 1\\
  1 + B\Delta t - \theta B\Delta t &\leq 1 - \theta B\Delta t\\
  1 + B\Delta t&\leq 1\\
  B\Delta t \leq 0,
\end{align*}
which also holds because $B<0$. The divergence problem appears only when $C < -1$, which is taken care of by the first inequality.
\section{Which one of the three schemes will you recommend for solving this equation with a) large time steps and b) small time steps?}

\section{The equation above is not a good model if $\rho vr/\mu$ is much greater than 1, which is the case for a body that is not very small. How can the model be extended to cover this case? Suggest a numerical scheme for the modified equation.}
This quantity is the Reynold number,
\begin{align*}
  \text{Re} = \frac{\rho v r}{\mu}
\end{align*}
and as mentioned earlier, the Stokes drag is valid only in the low Reynolds number regime. For large objects, we want a quadratic air resistance term. Instead of the Stokes drag, we can use 
\begin{align*}
  F_d &= -\frac{1}{2}C_D \rho A |v|v
\end{align*}
where $C_D$ is a dimensionless drag coefficient depending on the body's shape, A is the cross-sectional area. Newton's 2nd law looks like
\begin{align*}
  \dot v &= {g(\rho_f - \rho) \over \rho} -\frac{C_DA}{2V}|v|v = A + B|v|v,
\end{align*}
a non linear equation for $v$.
\subsection{Crank-Nicolson}
We can make a CN-scheme for this equation
\begin{align*}
  {v^{n+1} - v^n \over \Delta t} &= A + B|v^n|v^{n+1},
\end{align*}
where we instead of the arithmetic mean have used the geometric mean. Solving for $v^{n+1}$ gives
\begin{align*}
  v^{n+1} = {A\Delta t + v^n \over 1 + B\Delta t|v^n|}.
\end{align*}
\section{Suppose the shape of the body is much more complicated than a sphere so that simple fluid resistance formulas are too inaccurate. Explain briefly how one can compute (in principle) a more accurate drag force on the body from a detailed flow calculation with PDEs.}



\end{document}
